% Created 2022-02-18 Fr 09:18
% Intended LaTeX compiler: pdflatex
\documentclass[presentation]{beamer}
\usepackage[utf8]{inputenc}
\usepackage[T1]{fontenc}
\usepackage{graphicx}
\usepackage{grffile}
\usepackage{longtable}
\usepackage{wrapfig}
\usepackage{rotating}
\usepackage[normalem]{ulem}
\usepackage{amsmath}
\usepackage{textcomp}
\usepackage{amssymb}
\usepackage{capt-of}
\usepackage{hyperref}
\usepackage{minted}
\usetheme{default}
\author{Sebastian Stabinger}
\date{\today}
\title{Klasse zum Konvertieren von Temperaturen}
\hypersetup{
 pdfauthor={Sebastian Stabinger},
 pdftitle={Klasse zum Konvertieren von Temperaturen},
 pdfkeywords={},
 pdfsubject={},
 pdfcreator={Emacs 27.2 (Org mode 9.4.4)}, 
 pdflang={English}}
\begin{document}

\maketitle

\begin{frame}[label={sec:org706ae3b}]{Angabe}
Implementieren Sie eine Klasse welche zwischen \alert{Fahrenheit}, \alert{Celsius}
und \alert{Kelvin} konvertieren kann und umgekehrt.
\begin{block}{Funktionalität}
Sie sollen die Temperatur der Klasse in Fahrenheit, Celsius und Kelvin
setzten können. Sie sollen die Temperatur in Fahrenheit, Celsius und
Kelvin auslesen können.
\end{block}
\begin{block}{Frage}
Welchen Vorteil hat diese Vorgehensweise gegenüber der Definition von
Funktionen zum Konvertieren?
\end{block}
\begin{block}{Konvertierungsfunktionen}
\(Kelvin = (Fahrenheit + 459.67)\cdot\frac{5}{9}\)

\(Fahrenheit = Kelvin \cdot\frac{9}{5} - 459.67\)

\(Kelvin = Celsius + 273.15\)

\(Celsius = Kelvin - 273.15\)
\end{block}
\end{frame}
\begin{frame}[label={sec:org40f99e4},fragile]{Lösung}
 Generell werden wir das Problem folgendermaßen lösen: Beim setzen der
Temperatur in einer bestimmten Einheit rechnen wir diese sofort in
Kelvin um und speichern gleich die Temperatur in Kelvin. Beim Abfragen
der Temperatur müssen wir diese nur noch von Kelvin in die gewünschte
Einheit umrechnen. Dadurch benötigen wir nur eine Variable in der
Klasse und jeweils eine Funktion zum setzen und auslesen der
Temperatur pro Einheit.

Wir starten mit der Definition der Klasse.
\begin{minted}[numberblanklines=false,fontsize=\scriptsize]{c++}
class Tempconv {
\end{minted}

\begin{block}{Variablen}
Wir wollen die aktuelle Temperatur der Klasse in Kelvin speichern und
sinnvollerweise wählen wir {\color{solarizedYellow}\verb!double!} als Typ. Von Außerhalb der Klasse
soll kein direkter Zugriff auf diese Variable möglich sein, daher
definieren wir sie im {\color{solarizedYellow}\verb!private!} Bereich der Klasse. Variablen im
{\color{solarizedYellow}\verb!private!} Bereich beginnen üblicherweise mit einem Unterstrich um sie
als private zu kennzeichnen.
\begin{minted}[numberblanklines=false,fontsize=\scriptsize]{c++}
private:
  double _kelvin;
\end{minted}

Alle folgenden Dinge der Klasse sollen von Außerhalb zugänglich sein.
Wir starten daher den {\color{solarizedYellow}\verb!public!} Bereich der Klasse.
\begin{minted}[numberblanklines=false,fontsize=\scriptsize]{c++}
public:
\end{minted}
\end{block}

\begin{block}{Konstruktoren}
Wenn wir eine neue Instanz der Klasse erzeugen (z.B. eine Variable
dieser Klasse definieren) wollen wir, dass die Temperatur der Klasse
Null Grad Kelvin beträgt. Wir schreiben daher einen
Standardkonstruktor welcher {\color{solarizedYellow}\verb!_kelvin!} auf {\color{solarizedYellow}\verb!0!} setzt. Zur Erinnerung:
Ein Konstruktor ist eine Memberfunktion welche den \alert{gleichen Namen}
wie die Klasse und \alert{keinen Rückgabetyp} hat. Der Standardkonstruktor
hat zusätzlich eine \alert{leere Parameterliste}.
\begin{minted}[numberblanklines=false,fontsize=\scriptsize]{c++}
Tempconv() { _kelvin = 0; }
\end{minted}
\end{block}

\begin{block}{Memberfunktionen}
Im nächsten Schritt definieren wir die ganzen Funktionen der Klasse
(sogenannte Memberfunktionen).

\begin{block}{Setzen der Temperatur}
In den Funktionen zum setzen der Temperatur rechnen wir die übergebene
Temperatur nach Kelvin um und speichern diesen Wert in der privaten
Variable {\color{solarizedYellow}\verb!_kelvin!}.
\begin{block}{Setzen der Temperatur in Kelvin}
Da die Temperatur bereits in der korrekten Einheit übergeben wurde
müssen wir nichts umrechnen.
\begin{minted}[numberblanklines=false,fontsize=\scriptsize]{c++}
void set_kelvin(double kelvin) { _kelvin = kelvin; }
\end{minted}
\end{block}
\begin{block}{Setzen der Temperatur in Celsius}
Wir verwenden die Umrechnungsfunktion von Oben: \(Kelvin = Celsius + 273.15\)
\begin{minted}[numberblanklines=false,fontsize=\scriptsize]{c++}
void set_celsius(double celsius) { _kelvin = celsius + 273.15; }
\end{minted}
\end{block}
\begin{block}{Setzen der Temperaturen in Fahrenheit}
Wir verwenden die Umrechnungsfunktion von Oben: \(Kelvin = (Fahrenheit + 459.67)\cdot\frac{5}{9}\)
\begin{minted}[numberblanklines=false,fontsize=\scriptsize]{c++}
void set_fahrenheit(double fahrenheit) {
  _kelvin = (fahrenheit + 459.67) * (5.0 / 9.0);
}
\end{minted}
\end{block}
\end{block}
\begin{block}{Auslesen der Temperatur}
In den Funktionen zum Auslesen der Temperatur konvertieren wir die in
Kelvin gespeicherte Temperatur in die jeweils geforderte Einheit und
liefern das Ergebnis der Konvertierung als Ergebnis der Funktion
zurück.
\begin{block}{Ausgeben der Temperatur in Kelvin}
Da die Temperatur bereits in Kelvin gespeichert ist brauchen wir
nichts weiter zu machen als den Inhalt von {\color{solarizedYellow}\verb!_kelvin!} zurückzuliefern.
\begin{minted}[numberblanklines=false,fontsize=\scriptsize]{c++}
double get_kelvin() { return _kelvin; }
\end{minted}
\end{block}
\begin{block}{Ausgeben der Temperatur in Celsius}
Wir verwenden die Umrechnungsfunktion von Oben: \(Celsius = Kelvin - 273.15\)
\begin{minted}[numberblanklines=false,fontsize=\scriptsize]{c++}
double get_celsius() { return _kelvin - 273.15; }
\end{minted}
\end{block}
\begin{block}{Ausgeben der Temperatur in Celsius}
Wir verwenden die Umrechnungsfunktion von Oben: \(Fahrenheit = Kelvin \cdot\frac{9}{5} - 459.67\)
\begin{minted}[numberblanklines=false,fontsize=\scriptsize]{c++}
double get_fahrenheit() {
  return _kelvin * (9.0 / 5.0) - 459.67;
}
\end{minted}
\end{block}
\end{block}
\end{block}
\begin{block}{Main}
Damit haben wir die Klasse zum Konvertieren fertig geschrieben und
müssen nur noch eine {\color{solarizedYellow}\verb!main!} Funktion schreiben um die Funktionalität
der neuen Klasse testen.

Wir erzeugen eine neue Instanz unserer Klasse und testen ob die
anfängliche Temperatur tatsächlich Null Grad Kelvin beträgt.
\begin{minted}[numberblanklines=false,fontsize=\scriptsize]{c++}
Tempconv a;
cout << "Anfang:" << endl;
cout << "Temp: " << a.get_kelvin() << "K" << endl;
cout << "Temp: " << a.get_celsius() << "C" << endl;
cout << "Temp: " << a.get_fahrenheit() << "F" << endl << endl;
\end{minted}
-40 Fahrenheit und -40 Grad Celsius sollte etwa die gleiche Temperatur
 sein. Überprüfen wir das!
\begin{minted}[numberblanklines=false,fontsize=\scriptsize]{c++}
a.set_celsius(-40);
cout << "-40C = " << a.get_fahrenheit() << "F" << endl;
\end{minted}
Als Ausgaben bekommen wir folgendes. Die Klasse scheint also korrekt
zu funktionieren.
\begin{minted}[numberblanklines=false,fontsize=\scriptsize]{text}
Anfang:
Temp: 0K
Temp: -273.15C
Temp: -459.67F

-40C = -40F
\end{minted}
\end{block}
\end{frame}
\begin{frame}[label={sec:orgcc425ed},fragile]{Kompletter Sourcecode}
 \begin{minted}[numberblanklines=false,fontsize=\scriptsize]{c++}
#include <iostream>
using namespace std;

class Tempconv {
private:
  double _kelvin;

public:
  Tempconv() { _kelvin = 0; }  // Standardkonstruktor

  // Memberfunktionen zum Setzen der Temperatur
  void set_kelvin(double kelvin) { _kelvin = kelvin; }
  void set_celsius(double celsius) { _kelvin = celsius + 273.15; }
  void set_fahrenheit(double fahrenheit) {
    _kelvin = (fahrenheit + 459.67) * (5.0 / 9.0);
  }

  // Memberfunktionen zum Auslesen der Temperatur
  double get_kelvin() { return _kelvin; }
  double get_celsius() { return _kelvin - 273.15; }
  double get_fahrenheit() {
    return _kelvin * (9.0 / 5.0) - 459.67;
  }
}; // Ende der Tempconv-Klasse

int main() {
  Tempconv a;
  cout << "Anfang:" << endl;
  cout << "Temp: " << a.get_kelvin() << "K" << endl;
  cout << "Temp: " << a.get_celsius() << "C" << endl;
  cout << "Temp: " << a.get_fahrenheit() << "F" << endl << endl;

  a.set_celsius(-40);
  cout << "-40C = " << a.get_fahrenheit() << "F" << endl;
}
\end{minted}
\end{frame}
\end{document}
