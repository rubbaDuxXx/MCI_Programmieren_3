% Created 2018-05-04 Fri 19:50
\documentclass[presentation]{beamer}
\usepackage[utf8]{inputenc}
\usepackage[T1]{fontenc}
\usepackage{fixltx2e}
\usepackage{graphicx}
\usepackage{longtable}
\usepackage{float}
\usepackage{wrapfig}
\usepackage{rotating}
\usepackage[normalem]{ulem}
\usepackage{amsmath}
\usepackage{textcomp}
\usepackage{marvosym}
\usepackage{wasysym}
\usepackage{amssymb}
\usepackage{hyperref}
\tolerance=1000
\usepackage{minted}
\usetheme{default}
\author{Sebastian Stabinger}
\date{\today}
\title{upcast\_param}
\hypersetup{
  pdfkeywords={},
  pdfsubject={},
  pdfcreator={Emacs 25.3.1 (Org mode 8.2.10)}}
\begin{document}

\maketitle
\begin{frame}{Outline}
\tableofcontents
\end{frame}

\#include <iostream>
using namespace std;

class Basis \{
public:
  int a;
  void print() \{ cout << "Basisklasse mit Nummer " << a << endl; \}
\};

class Abgeleitet : public Basis \{
public:
  void print() \{ cout << "Abgeleitete Klasse mit Nummer " << a << endl; \}
\};

void callmyprint(Basis \&b) \{ b.print(); \}

int main() \{
  Abgeleitet ab;
  ab.a = 42;
  ab.print(); \emph{/ Abgeleitete Klasse mit Nummer 42
  callmyprint(ab); /} Basisklasse mit Nummer 42
  // ab wird implizit in Basis konvertiert
\}
% Emacs 25.3.1 (Org mode 8.2.10)
\end{document}