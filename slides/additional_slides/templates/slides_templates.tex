% Created 2022-02-18 Fr 10:39
% Intended LaTeX compiler: pdflatex
\documentclass[presentation]{beamer}
\usepackage[utf8]{inputenc}
\usepackage[T1]{fontenc}
\usepackage{graphicx}
\usepackage{grffile}
\usepackage{longtable}
\usepackage{wrapfig}
\usepackage{rotating}
\usepackage[normalem]{ulem}
\usepackage{amsmath}
\usepackage{textcomp}
\usepackage{amssymb}
\usepackage{capt-of}
\usepackage{hyperref}
\usepackage{minted}
\usepackage[utf8]{inputenc}
\usepackage{color}
\usetheme[height=7mm]{Rochester}
\setbeamertemplate{footline}[frame number]
\usecolortheme[accent=red, light]{solarized}
\setbeamercolor{frametitle}{bg=solarizedRebase02,fg=solarizedAccent}
\setbeamercolor{author in head/foot}{bg=solarizedRebase02,fg=solarizedRebase01}
\setbeamercolor{title in head/foot}{bg=solarizedRebase02,fg=solarizedRebase01}
\setbeamercolor{block title}{bg=solarizedRebase0,fg=solarizedRebase02}
\setbeamercolor{block body}{bg=solarizedRebase02,fg=solarizedRebase0}
\setbeamercolor{item}{bg=solarizedRebase02,fg=solarizedAccent}
\beamertemplatenavigationsymbolsempty
\usemintedstyle{manni}
\AtBeginSection[]{
\begin{frame}
\vfill
\centering
\begin{beamercolorbox}[sep=8pt,center,shadow=true,rounded=true]{title}
\Huge\insertsectionhead\par%
\end{beamercolorbox}
\vfill
\end{frame}
}
\usetheme{default}
\author{Thomas Hausberger, Matthias Panny \\ Ursprünglich erstellt von Sebastian Stabinger}
\date{SS2022}
\title{Templates}
\hypersetup{
 pdfauthor={Thomas Hausberger, Matthias Panny \\ Ursprünglich erstellt von Sebastian Stabinger},
 pdftitle={Templates},
 pdfkeywords={},
 pdfsubject={},
 pdfcreator={Emacs 27.2 (Org mode 9.4.4)}, 
 pdflang={Ger}}
\begin{document}

\maketitle

\section{Problem}
\label{sec:orgf3fb823}
\begin{frame}[label={sec:org0843f1f},fragile]{Quadrieren von Zahlen in C}
 \begin{exampleblock}{Beispiel}
\begin{minted}[numberblanklines=false,fontsize=\scriptsize]{c}
short quad_short(short a) { return a * a; }
int quad_int(int a) { return a * a; }
float quad_float(float a) { return a * a; }
double quad_double(double a) { return a * a; }

#include <stdio.h>

int main() {
  printf("Int: %d\n", quad_int(23));
  printf("Float: %f\n", quad_float(24.12));
}
\end{minted}
\end{exampleblock}
\begin{itemize}
\item Für jeden Datentyp brauchen wir eine eigene Funktion
\item Die Funktionen müssen alle unterschiedlich benannt sein
\item Aber eigentlich machen wir \alert{in jeder Funktion das selbe}!
\end{itemize}
\end{frame}
\begin{frame}[label={sec:org1c1d5cc},fragile]{Quadrieren von Zahlen in C++}
 \begin{itemize}
\item Bis jetzt haben wir \alert{Funktionsüberladung} kennen gelernt. Das macht
die Situation zumindest in der Verwendung der Funktionen besser: Wir
können \alert{alle Funktionen gleich benennen}
\end{itemize}
\begin{exampleblock}{Beispiel}
\begin{minted}[numberblanklines=false,fontsize=\scriptsize]{c++}
#include <iostream>
using namespace std;

short quad(short a) { return a * a; }
int quad(int a) { return a * a; }
float quad(float a) { return a * a; }
double quad(double a) { return a * a; }

int main() {
  cout << "Int: " << quad(23) << endl;
  cout << "Float: " << quad(24.12) << endl;
}
\end{minted}
\end{exampleblock}
\begin{itemize}
\item Wir machen \alert{immer noch das selbe} in allen Funktionen!
\end{itemize}
\end{frame}
\section{Lösung: Templates}
\label{sec:orgb8048a7}
\begin{frame}[label={sec:orgeb7bdab},fragile]{Grundlagen}
 \begin{itemize}
\item Wir können mittels Templates \alert{einen Typ als Parameter} übergeben
\item Möglich bei \alert{Funktionen} oder \alert{Klassen}
\item Geschieht mittels {\color{solarizedYellow}\verb!template <typename T>!} vor der Funktion oder Klasse
\item T ist dann ein \alert{Platzhalter} für einen beliebigen Typ den wir übergeben können
\item Templateparameter werden in spitzen Klammern statt runder Klammern
übergeben. z.B. {\color{solarizedYellow}\verb!quad<int>(i)!} um einer Templatefunktion {\color{solarizedYellow}\verb!quad!} als
Typ {\color{solarizedYellow}\verb!int!} und als normalen Parameter {\color{solarizedYellow}\verb!i!} zu übergeben.
\end{itemize}
\begin{block}{Hinweis}
Aus historischen Gründen kann man statt {\color{solarizedYellow}\verb!template <typename T>!} auch
{\color{solarizedYellow}\verb!template <class T>!} schreiben.
\end{block}
\end{frame}
\begin{frame}[label={sec:orgdd2dfc0},fragile]{Funktionstemplates}
 Funktionstemplates werden verwendet um bei einer Funktion einen Typ
als Parameter übergeben zu können
\begin{exampleblock}{Beispiel für {\color{solarizedYellow}\textt{quad}}}
\begin{minted}[numberblanklines=false,fontsize=\scriptsize]{c++}
#include <iostream>
using namespace std;

template <typename T> T quad(T a) { return a * a; }

int main() {
  cout << "Int: " << quad<int>(23) << endl;
  cout << "Float: " << quad<float>(24.12) << endl;
}
\end{minted}
\end{exampleblock}
\begin{itemize}
\item Aufruf: {\color{solarizedYellow}\verb!f< Templateparameter >( Normale Parameter )!}
\end{itemize}
\end{frame}
\begin{frame}[label={sec:org6607fd9},fragile]{Was passiert hinter den Kulissen?}
 \begin{itemize}
\item Wir haben {\color{solarizedYellow}\verb!quad!} folgendermaßen definiert:
\end{itemize}
\begin{minted}[numberblanklines=false,fontsize=\scriptsize]{c++}
template <typename T> T quad(T a) { return a * a; }
\end{minted}
\begin{itemize}
\item Irgendwo im Code rufen wir {\color{solarizedYellow}\verb!quad<int>(42);!} auf. Der Compiler
\alert{erzeugt automatisch} eine \alert{Funktion mit konkretem Typ} mit Hilfe
unseres Funktionstemplates:
\end{itemize}
\begin{minted}[numberblanklines=false,fontsize=\scriptsize]{c++}
template <typename T> T quad(T a) { return a * a; }
// T wird durch quad<int>(42) der Typ int zugewiesen --->
int quad(int a) { return a * a; }
// Diese Funktion wird dann ganz normal auf den Wert 42 angewendet
\end{minted}
\begin{itemize}
\item Templatefunktionen sind daher bei der Ausführung nicht langsamer als
"ganz normale" Funktionen
\item Beim Compilieren wird also tatsächlich \alert{neuer Code erzeugt}
\end{itemize}
\begin{alertblock}{Hinweis}
Template kann man als \alert{Schablone} ins Deutsche übersetzten: Wir haben
eine Schablone für Funktionen und Klassen mit deren Hilfe wir
spezifische Versionen von Funktionen und Klassen erzeugen können.
\end{alertblock}
\end{frame}
\begin{frame}[label={sec:org58caafc},fragile]{Typinferenz}
 \begin{itemize}
\item Wir haben {\color{solarizedYellow}\verb!quad!} folgendermaßen definiert:
\end{itemize}
\begin{minted}[numberblanklines=false,fontsize=\scriptsize]{c++}
template <typename T> T quad(T a) { return a * a; }
\end{minted}
\begin{itemize}
\item Es ist offensichtlich etwas unpraktisch den gewünschten Typ bei
einer Templatefunktion immer explizit in spitzen Klammern angeben zu
müssen
\item Wenn wir {\color{solarizedYellow}\verb!quad<int>(42)!} aufrufen ist dem Compiler ja bereits durch
den Parameter {\color{solarizedYellow}\verb!42!} bekannt, dass {\color{solarizedYellow}\verb!T!} ein Integer sein muss
\item Tatsächlich müssen wir den Typ nicht explizit angeben, falls dem
Compiler der Typ bekannt sein kann (was bei Funktionstemplates oft
der Fall ist)
\item Wir können also einfach {\color{solarizedYellow}\verb!quad(42)!} schreiben
\end{itemize}
\end{frame}
\begin{frame}[label={sec:org989d6c4},fragile]{Beispiel {\color{solarizedYellow}\texttt{max}}}
 \begin{minted}[numberblanklines=false,fontsize=\scriptsize]{c++}
#include <iostream>
using namespace std;

template <typename T> T t_max(T a, T b) { // t_max weil es max schon gibt
  T ergebnis; // Wir koennen auch Variablen vom unbekannten Typ erzeugen
  if (a > b)
    ergebnis = a;
  else
    ergebnis = b;

  return ergebnis;
}

int main() {
  int a = 23, b = 42;
  double c = 34.2, d = 1.3;
  cout << "max int = " << t_max(a, b) << endl;
  cout << "max double = " << t_max(c, d) << endl;
  // Was passiert wenn wir Typen mixen?
  // cout << "max int/double = " << t_max(a, c) << endl;
}
\end{minted}
\end{frame}
\begin{frame}[label={sec:org077a881},fragile]{Beispiel {\color{solarizedYellow}\texttt{max}} --- Gemischte Typen}
 Was passiert beim Aufruf von Folgendem?
\begin{minted}[numberblanklines=false,fontsize=\scriptsize]{c++}
cout << "max int/double = " << t_max(a, c) << endl;
\end{minted}
\begin{block}{Ausgabe Compiler}
\begin{minted}[numberblanklines=false,fontsize=\scriptsize]{text}
max.cpp: In function ‘int main()’:
max.cpp:20:44: error: no matching function for call to 
	       ‘t_max(int&, double&)’
   cout << "max int/double = " << t_max(a, c) << endl;
					    ^
max.cpp:20:44: note: candidate is:
max.cpp:4:25: note: template<class T> T t_max(T, T)
 template <typename T> T t_max(T a, T b) {
			 ^
max.cpp:4:25: note:   template argument deduction/substitution failed:
max.cpp:20:44: note:   deduced conflicting types for parameter ‘T’ 
		       (‘int’ and ‘double’)
   cout << "max int/double = " << t_max(a, c) << endl;
					    ^
\end{minted}
\end{block}
\end{frame}
\begin{frame}[label={sec:org06f8e2c},fragile]{Klassentemplates --- Beispiel}
 \begin{itemize}
\item Funktionieren genauso wie Funktionstemplates. Man kann Typen als
Parameter für \alert{eine Instanz einer Klasse} übergeben
\end{itemize}
\begin{minted}[numberblanklines=false,fontsize=\scriptsize]{c++}
#include <iostream>
using namespace std;

template <typename Sum, typename Count> class average {
private:
  Sum _sum = 0;
  Count _count = 0;

public:
  void add(Sum val) { _sum += val; _count++; }

  Sum avg() { if (_count == 0) return 0; else return _sum / _count; }
};

int main() {
  average<float, int> a;
  cout << "a " << a.avg() << endl;
  a.add(23);
  a.add(27);
  cout << "a " << a.avg() << endl;
}
\end{minted}
\end{frame}
\begin{frame}[label={sec:orgf3070fc},fragile]{Abschluss}
 \begin{itemize}
\item Wie bereits gesagt kann man sich Templates als \alert{Schablonen}
vorstellen. In diesen Schablonen werden Platzhalter für Typen durch
einen übergebenen Typer ersetzt
\item Beim Compilieren wird tatsächlich \alert{neuer Code generiert}. Das führt
dazu, dass die Compilezeiten länger werden.
\item Wir haben Templates z.B. schon bei {\color{solarizedYellow}\verb!vector!} gesehen. Der zu
speichernde Typ wird in spitzen Klammern angegeben:
\end{itemize}
\begin{minted}[numberblanklines=false,fontsize=\scriptsize]{c++}
vector<int> v; v.push_back(23);
\end{minted}
\begin{block}{Hinweis}
Templates sind bei weitem komplexer und mächtiger als es hier den
Anschein hat. Es handelt sich im Prinzip um eine eigene
Programmiersprache (das war eigentlich nicht so beabsichtigt).

Wer sich den Wahnsinn des "\alert{Template Metaprogrammings}" anschauen will
kann z.B. auf der \uline{\href{https://en.wikipedia.org/wiki/Template\_metaprogramming}{\color{blue}{Wikipedia}}} Seite anfangen.
\end{block}
\end{frame}
\end{document}
